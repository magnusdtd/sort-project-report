\section{Kết luận}

Báo cáo đã trình bày và phân tích chi tiết về các thuật toán sắp xếp khác nhau, bao gồm Bubble Sort, Insertion Sort, Selection Sort, Heap Sort, Merge Sort, Quick Sort, Counting Sort, Radix Sort, và Flash Sort. Mỗi thuật toán được mô tả với các bước thực hiện cụ thể và minh họa bằng các ví dụ trực quan.

Thực nghiệm đã được tiến hành, so sánh thời gian chạy, đếm số phép so sánh của các thuật toán trên các bộ dữ liệu khác nhau: dữ liệu ngẫu nhiên, dữ liệu gần như đã sắp xếp, dữ liệu đã sắp xếp, và dữ liệu sắp xếp ngược. Kết quả thực nghiệm được trình bày chi tiết trong các bảng và biểu đồ, cho thấy sự khác biệt về hiệu suất của từng thuật toán trong các trường hợp khác nhau.

Qua quá trình thực nghiệm, nhận thấy rằng:
\begin{itemize}
    \item Các thuật toán sắp xếp đơn giản như Bubble Sort, Insertion Sort và Selection Sort có thời gian chạy chậm hơn so với các thuật toán phức tạp hơn như Heap Sort, Merge Sort và Quick Sort.
    \item Nhóm các thuật toán cơ bản (Selection Sort, Insertion Sort, Bubble Sort, Shaker Sort) có hiệu suất không ổn định qua các trường hợp của dữ liệu. Đặc biệt, hiệu suất của Selection Sort có hiệu xuất rất kém.
    \item Flash Sort và Radix Sort cho thấy hiệu suất vượt trội trong một số trường hợp nhất định, đặc biệt là với các bộ dữ liệu lớn.
    \item Merge Sort và Quick Sort là hai thuật toán có hiệu suất ổn định và thời gian chạy tốt nhất trên hầu hết các bộ dữ liệu.
    \item Counting Sort và Radix Sort có hiệu suất tốt và ổn định nhưng hai thuật toán này chỉ sử dụng được trên kiểu dữ liệu là số nguyên.
    \item item Nhóm thuật toán cải tiến (Shell Sort, Heap Sort, Merge Sort, Quick Sort) vẫn chưa thể hiện được sự khác nhau rõ ràng qua thực nghiệm.
\end{itemize}

Báo cáo này không chỉ giúp hiểu rõ hơn về các thuật toán sắp xếp mà còn cung cấp cái nhìn tổng quan về cách lựa chọn thuật toán phù hợp cho từng loại dữ liệu cụ thể. Hy vọng rằng những kết quả và phân tích trong báo cáo này sẽ là tài liệu tham khảo hữu ích cho các bạn sinh viên và những người quan tâm đến lĩnh vực cấu trúc dữ liệu và giải thuật.
