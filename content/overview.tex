\section{Thông tin chung}

\subsection{Giới thiệu sơ lược các thuật toán}

Thuật toán sắp xếp đóng vai trò quan trọng trong khoa học máy tính và các ứng dụng thực tế. Bài báo cáo này tập trung nghiên cứu và so sánh 11 thuật toán sắp xếp phổ biến, bao gồm: Selection Sort, Insertion Sort, Bubble Sort, Shaker Sort, Shell Sort, Heap Sort, Merge Sort, Quick Sort, Counting Sort, Radix Sort, và Flash Sort. Các thuật toán được phân loại thành ba nhóm: thuật toán cơ bản (Selection Sort, Insertion Sort, Bubble Sort, Shaker Sort), thuật toán cải tiến (Shell Sort, Heap Sort, Merge Sort, Quick Sort), và thuật toán không so sánh (Counting Sort, Radix Sort, Flash Sort). Đặc điểm chính, thời gian chạy thực tế và độ phức tạp thời gian của từng thuật toán được phân tích, từ các thuật toán có độ phức tạp $O(n^2)$ đến các thuật toán tối ưu hơn với $O(n \log n)$ hoặc $O(n)$. Kết quả nghiên cứu cung cấp cái nhìn tổng quan về ưu, nhược điểm của từng thuật toán, làm cơ sở cho việc lựa chọn thuật toán phù hợp với các bài toán sắp xếp trong thực tiễn.

\subsection{Dữ liệu}

Dữ liệu sử dụng trong thí nghiệm là mảng số nguyên với các kích thước khác nhau: 10000, 30000, 50000, 100000, 300000, 500000. Tương ứng từng kích thước có 4 kiểu thứ tự của dữ liệu: Mảng được sắp xếp, mảng gần được sắp xếp hoàn chỉnh, mảng được sắp xếp ngược, mảng có thứ tự ngẫu nhiên. Các hàm phát sinh dãy số được giáo viên hướng dẫn thực hành cung cấp trong file DataGenerator.cpp. Các mảng số nguyên phát sinh được yêu cầu sắp xếp tăng dần bằng các thuật toán đề cập trong thí nghiệm.

\subsection{Cấu hình thực nghiệm}
Số liệu thực nghiệm trong báo cáo có được thông qua việc sử dụng GitHub Actions để chạy thực nghiệm. Cấu hình của máy ảo cho một public repository của \href{https://docs.github.com/en/actions/using-github-hosted-runners/using-github-hosted-runners/about-github-hosted-runners#supported-runners-and-hardware-resources}{GitHub}:
\begin{itemize}
    \item CPU: 4 Processor
    \item RAM: 16GB
    \item Storage: 14GB
    \item OS: Ubuntu 24.04.1 LTS
\end{itemize}
 

 
